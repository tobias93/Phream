%!TEX root = ../documentation.tex

\documentclass[%
	pdftex,
	oneside,			% Einseitiger Druck.
	11pt,				% Schriftgroesse
	parskip=half,		% Halbe Zeile Abstand zwischen Absätzen.
	headsepline,		% Linie nach Kopfzeile.
	footsepline,		% Linie vor Fusszeile.
	abstracton,			% Abstract Überschriften
	listof=totoc,
	toc=bibliography,
	headings=optiontohead,
	plainfootsepline
]{scrreprt}

%% Schrift -> Muss hier bleiben sonst gibts Probleme mit dem Deckblatt
\usepackage{xstring}
\usepackage[utf8]{inputenc}
\usepackage[T1]{fontenc}
\usepackage{lmodern}
\renewcommand{\familydefault}{\sfdefault} % serifen lose Schriftart

% Sprache
\usepackage[ngerman]{babel}

%% Mathepakete & Schrift
\usepackage{amsmath}
\usepackage{amsthm}
\usepackage{amsbsy}
\usepackage{amssymb}
\usepackage{sansmath}
\sansmath

%% Für Tables/Images/Figures
\usepackage{here}



%% ---------------------- Allgemeine Einstellungen ----------------------

\newcommand{\kurs}{TINF14-AIBC}

\newcommand{\titel}{Entwicklung einer Foto-App für Android} % Titel noch im Beta-Status
%\newcommand{\titelA}{--} % Für längere Titel
%\newcommand{\titelB}{--}


\newcommand{\datumAbgabe}{November 2015}

\newcommand{\abgabeort}{Mannheim}
\newcommand{\studiengang}{Studiengang Angewandte Informatik}
\newcommand{\dhbw}{Mannheim}

\newcommand{\zeitraum}{4 Wochen}
\newcommand{\arbeit}{Studienarbeit}
\newcommand{\autor}{Tobias Dorra und Philipp Pütz}

\newcommand{\sprache}{de}

\newcommand{\spaltenabstand}{10pt}
\newcommand{\zeilenabstand}{1.5}

\newcommand{\artikelstudiengang}{im}
\newcommand{\anderdh}{an der Dualen Hochschule Baden-Württemberg}
\newcommand{\von}{von}

\newcommand{\sperrvermerk}{Sperrvermerk}
\newcommand{\erklaerung}{Erklärung}
\newcommand{\abkverz}{Abkürzungsverzeichnis}
\newcommand{\glossar}{Glossar}


%%%%%%%%%%%%%%%%%%%%%%%%%%%%%%%%%%%%%%%%%%%%%%%%%%%%%%%%%%%%%%%%%%%%%%%%%%%%%%%%


%%%%%%% Package Includes %%%%%%%

\usepackage[a4paper, left=25mm,right=25mm, top=38mm, bottom=43mm]{geometry}
\usepackage[activate]{microtype} %Zeilenumbruch und mehr
\usepackage[onehalfspacing]{setspace}
\usepackage{makeidx}
\usepackage[autostyle=true,german=quotes]{csquotes}
\usepackage{longtable}
\usepackage{graphicx}
\usepackage{calc}		%zum Rechnen (Bildtabelle in Deckblatt)
\usepackage{wrapfig}
\usepackage[perpage, hang, multiple, stable]{footmisc} % Bibverzeichnis
\usepackage[printonlyused,footnote]{acronym}
\usepackage{textcomp}
\usepackage{tabularx}
\usepackage{ltablex}
\usepackage{booktabs}
\usepackage{longtable}

 % Für Codebeispiele
\usepackage{listings}

\usepackage{epstopdf} % Eps Grafikeinbidnung
\usepackage{nameref}

%% Paket um Textteile drehen zu können
\usepackage{rotating}

%% Paket um Seite im Querformat anzuzeigen
\usepackage{lscape}

%% Farben (Angabe in HTML-Notation mit großen Buchstaben)
\usepackage{xcolor} 	%xcolor für HTML-Notation
\definecolor{LinkColor}{HTML}{000000}
\definecolor{ListingBackground}{HTML}{FCF7DE}


%% Programmiersprachen Highlighting (Listings)
\newcommand{\listingsettings}{%
	\lstset{%
		language=Java,			% Standardsprache des Quellcodes
		%numbers=left,			% Zeilennummern links
		stepnumber=1,			% Jede Zeile nummerieren.
		numbersep=5pt,			% 5pt Abstand zum Quellcode
		numberstyle=\tiny,		% Zeichengrösse 'tiny' für die Nummern.
		breaklines=true,		% Zeilen umbrechen wenn notwendig.
		breakautoindent=true,	% Nach dem Zeilenumbruch Zeile einrücken.
		postbreak=\space,		% Bei Leerzeichen umbrechen.
		tabsize=2,				% Tabulatorgrösse 2
		basicstyle=\ttfamily\footnotesize, % Nichtproportionale Schrift, klein für den Quellcode
		showspaces=false,		% Leerzeichen nicht anzeigen.
		showstringspaces=false,	% Leerzeichen auch in Strings ('') nicht anzeigen.
		extendedchars=true,		% Alle Zeichen vom Latin1 Zeichensatz anzeigen.
		captionpos=b,			% sets the caption-position to bottom
		backgroundcolor=\color{ListingBackground}, % Hintergrundfarbe des Quellcodes setzen.
		xleftmargin=0pt,		% Rand links
		xrightmargin=0pt,		% Rand rechts
		frame=single,			% Rahmen an
		frameround=ffff,
		rulecolor=\color{darkgray},	% Rahmenfarbe
		fillcolor=\color{ListingBackground}
	}
}



%%%%%% Configuration %%%%%
% Titel, Autor und Datum

\title{\titel}
\author{\autor}
\date{\datum}

% PDF Einstellungen
\usepackage[%
	pdftitle={\titel},
	pdfauthor={\autor},
	pdfsubject={\arbeit},
	pdfcreator={pdflatex, LaTeX with KOMA-Script},
	pdfpagemode=UseOutlines, 		% Beim Oeffnen Inhaltsverzeichnis anzeigen
	pdfdisplaydoctitle=true, 		% Dokumenttitel statt Dateiname anzeigen.
	pdflang={\sprache}, 			% Sprache des Dokuments.
]{hyperref}

% (Farb-)einstellungen für die Links im PDF
\hypersetup{%
	colorlinks=true, 		% Aktivieren von farbigen Links im Dokument
	linkcolor=LinkColor, 	% Farbe festlegen
	citecolor=LinkColor,
	filecolor=LinkColor,
	menucolor=LinkColor,
	urlcolor=LinkColor,
	linktocpage=true, 		% Nicht der Text sondern die Seitenzahlen in Verzeichnissen klickbar
	bookmarksnumbered=true 	% Überschriftsnummerierung im PDF Inhalt anzeigen.
}

% Workaround um Fehler in Hyperref, muss hier stehen bleiben
\usepackage{bookmark} %nur ein latex-Durchlauf für die Aktualisierung von Verzeichnissen nötig

% Schriftart in Captions etwas kleiner
\addtokomafont{caption}{\small}


% Literaturverweise

\usepackage[
	backend=bibtex,		% empfohlen. Falls biber Probleme macht: bibtex
	bibwarn=true,
	bibencoding=ascii,	% wenn .bib in utf8, sonst ascii
	sortlocale=de_DE,
	style=numeric % 
	%isbn=false,
	%sorting=none,	%Zitierstil. Siehe http://ctan.mirrorcatalogs.com/macros/latex/contrib/biblatex/doc/biblatex.pdf
]{biblatex}


%% ---------------------- Literatur ----------------------
\bibliography{bibliographie}

\setlength{\bibitemsep}{10pt}
\setlength{\bibhang}{50pt}

\setcounter{biburllcpenalty}{7000}
\setcounter{biburlucpenalty}{8000}
\setcounter{biburlnumpenalty}{8000}

% Hurenkinder und Schusterjungen verhindern
% http://projekte.dante.de/DanteFAQ/Silbentrennung
\clubpenalty=10000
\widowpenalty=10000
\displaywidowpenalty=10000


\setlength{\tabcolsep}{\spaltenabstand}
\renewcommand{\arraystretch}{\zeilenabstand}

%% Header und Footer
\usepackage[autooneside=false,automark]{scrlayer-scrpage}
   
\clearpairofpagestyles
\cfoot*{\pagemark}
\ihead{\ifstr{\rightbotmark}{\leftmark}{}{\rightbotmark}}
\ohead{\leftmark}