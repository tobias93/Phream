%!TEX root = ../documentation.tex

\chapter{Einführung}

Bei gemeinsamen Freizeitaktivitäten mit Freunden entstehen oft viele Fotos. Diese werden dann entweder gar nicht oder erst Wochen später miteinander geteilt.

Unser Ziel ist es, die Benutzererfahrung beim Teilen der Bilder zu verbessern. Das soll mit einer eigenen Foto-App passieren. Anstatt die gesammelten Bilder erst nach der Veranstaltung auszuwählen und dann an die entsprechenden Kontakte zu schicken, sollen die geschossenen Bilder bereits während der Veranstaltung automatisch mit den anderen anwesenden Freunden geteilt werden.

\section{Funktionen}

\subsection{Organisation der Fotos}
Die Fotos werden in sogenannten \enquote{Streams} gruppiert. Jeder Stream repräsentiert dabei ein gemeinsames Erlebnis mit Freunden. Streams können umbenannt oder gelöscht werden. Beim Löschen eines Streams werden alle enthaltenen Bilder ebenfalls gelöscht.

Um einen Stream mit Fotos zu befüllen, kann entweder die Gerätekamera genutzt werden, oder es werden Fotos aus der Android-Galerie importiert. Die Fotos werden mit Name und Vorschaubild im Stream angezeigt. Außerdem existiert eine Detailansicht, in der man das Foto vergrößert betrachten kann. Fotos können umbenannt oder gelöscht werden. Ausgewählte Fotos können wieder zurück in die Android-Galerie exportiert, oder über die Teilen-Funktion anderen Apps zugänglich gemacht werden.

\subsection{Teilen von Fotos}
Zu einem Stream können mehrere Benutzer hinzugefügt werden. Die Fotos in einem Stream werden live zwischen allen Teilnehmern synchronisiert.

Alle Teilnehmer in einem Stream sind gleichberechtigt. Es gibt also keinen \enquote{Streamadministrator} mit besonderen Rechten. Jeder Teilnehmer eines Streams kann Bilder hinzufügen, Bilder löschen und die Teilnehmerliste verwalten. Insbesondere ist jeder Teilnehmer auch dazu berechtigt, weitere Teilnehmer hinzuzufügen.

Entfernt ein Benutzer einen Stream, so bleibt der Stream bei den anderen Teilnehmern noch vorhanden. Bei diesen wird lediglich der Benutzer aus der Teilnehmerliste entfernt.

Wird ein Benutzer nachträglich zu einem Stream hinzugefügt, so erhält er nicht nur die neuen Bilder, sondern bekommt auch automatisch Zugriff auf alle Bilder, die bereits vor seinem Eintritt im Stream vorhanden waren.

\subsection{Nutzung alternativer Methoden zur Datenübertragung}
Wenn man gezwungen ist, das Mobilfunknetz zu benutzen, ist die Internetverbindung oft schlecht. Außerdem haben viele Smartphone-Nutzer ein begrenztes Datenvolumen. Daher sollen die Daten nicht über das Internet, sondern über alternative Verbindungen übertragen werden. Diese beinhalten:

\begin{enumerate}
	\item Bluetooth
	\item Bluetooth Low Energy (eventuell, falls technisch machbar)
	\item Wi-Fi Direct
\end{enumerate}

\section{Projektumfang}

\subsection{Umgesetzte Funktionalitäten}

Die Anwendung, die im Rahmen dieses Projektes entwickelt wurde, stellt lediglich einen Prototyp dar. Es werden also nicht alle beschriebenen Funktionen umgesetzt. Bei der Architektur wurde jedoch darauf geachtet, dass die Anwendung um die beschriebenen Funktionalitäten erweiterbar ist. Die folgenden Funktionalitäten wurden im Prototyp umgesetzt:

\begin{center}
\begin{longtable}{|l|p{12cm}|}
\toprule
\textbf{ID} & \textbf{Beschreibung} \\
\hline
\endhead
\hline
\endfoot
F-10 & Bilder können zur App hinzufügt werden\\
F-10.1 & Bilder können über die Kamera aufgenommen werden\\
F-10.2 & Bilder aus der Galerie geladen werden\\
F-20 & Bilder können verwaltet werden\\
F-20.1 & Bilder können gelöscht und umbenannt werden\\
F-20.2 & Bilder haben verschiedene Attribute (Bildtitel, Aufnahmeinformationen)\\
F-30 & \enquote{Streams} (Ordner) gruppieren Bilder\\
F-30.1 & Streams können erstellt, gelöscht und umbenannt werden\\
F-40 & Bilder werden in einer listen ähnlichen Darstellung mit Vorschaubild und Bildinformationen angezeigt\\
F-50 & Bilder können in einer Vollbildansicht angezeigt werden\\
F-50.1 & Aus der Vollbildansicht können Bilder in die Galerie exportiert und per E-Mail bzw. zu teilen\\
F-60 & Die Architektur ist erweiterbar\\
\hline
D-10 & Alle Appdaten (Streams, Bilder, Datenbanken) werden in einen speziellen Verzeichnis gespeichert\\
D-10 & Gespeicherte Bilder erhalten als Zusatz das aktuelle Datum\\
D-20 & Alle Bild- und Streaminformationen werden in einer Datenbank gespeichert\\
\hline
UI-10 &  Es existiert eine Hauptansicht\\
UI-10.1 &  Die Hauptansicht zeigt Vorschaubilder von aufgenommene und aus der Galerie geladenen Bilder an\\
UI-10.2 &  Zu jedem Bild werden in der Hauptansicht der Bildtitel und Aufnahmeinformationen angezeigt\\
UI-10.3 &  Durch an tippen der Bilder startet eine neue Activity die das Bild in einer Vollbildansicht anzeigt\\
UI-10.4 &  Durch einen \enquote{long tap} auf ein Bild wir ein Kontext Menü angezeigt\\
UI-10.4.1 & Im Kontext Menü stehen folgende Optionen zur Verfügung: Bild umbenennen, Bild löschen, Bild in Galerie exportieren\\
UI-10.5 &  Ein Floating-Action Button Menü bietet weitere Optionen\\
UI-10.5.1 &  Über das Menü können Bilder mit der Kamera aufgenommen werden\\
UI-10.5.2 &  Über das Menü können Bilder aus der Galerie geladen werden\\
UI-10.6 & Ein \enquote{Navigation Drawer} bietet die Möglichkeit neue \enquote{Streams} anzulegen, die ähnlich wie Ordner, Bilder gruppieren\\
UI-10.6.1 & Die Hauptansicht zeigt den Inhalt eines Streams an\\
UI-10.7 & Über das Hauptmenü können Streams umbenannt werden\\
UI-10.8 & Über das Hauptmenü können Streams und alle zugehörigen Bilder gelöscht werden\\
UI-10.9 & Eine Standardansicht animiert den Nutzer zum anlegen eines Streams, wenn kein Stream angelegt ist\\
\hline
UI-20 &  Es existiert eine Vollbildansicht für Bilder\\
UI-20.1 & Die Vollbildansicht wird durch an tippen eines Bildes in der Hauptansicht geöffnet\\
UI-20.2 & Es ist möglich ein Bild aus der Vollbildansicht in die Galerie zu exportieren\\
UI-20.3 & Es ist möglich ein Bild aus der Vollbildansicht zu teilen (Versand per E-Mail falls E-Mail Client vorhanden)\\
\hline
\end{longtable}
\end{center}

\subsection{Offene Funktionalitäten}

Die im folgenden aufgelisteten Funktionalitäten wurden bisher nicht implementiert und werden für weitere Versionen vorgemerkt.

\begin{center}
\begin{longtable}{|l|p{12cm}|}
\toprule
\textbf{ID} & \textbf{Beschreibung} \\
\hline
\endhead
\hline
\endfoot
F-70 & Es können Benutzer zu Streams hinzugefügt werden\\
F-70.1 & Neue Benutzer erhalten Zugriff auf vorhandene und zukünftige Bilder im Stream\\
F-80 & Bilder werden zwischen den verschiedenen Nutzern live synchronisiert\\
F-90 & Es existiert ein Art Synchronisationslog der die unterschiedlichen Aktivitäten der Benutzer untereinander abgleicht\\
F-100 & Benutzer können aus Streams austreten \\
F-100.1 & Tritt ein Benutzer aus einem Stream aus, so werden lokal alle Bilder des Streams und der Stream selbst gelöscht\\
F-100.2 & Ausgetretene Benutzer werden bei anderen Benutzern aus der Streamteilnehmerliste entfernt\\
F-110 & Es stehen verschiedene Synchronisationsmöglichkeiten zur Verfügung:\\
F-110.1 & Bluetooth\\
F-110.2 & Bluetooth Low Energy (falls verfügbar)\\
F-110.3 & Wi-Fi Direct\\
F-120 & Alle Streamteilnehmer können über eine Teilnehmerliste eingesehen werden\\
F-130 & In der Vollbildansicht kann an Bilder herangezoomt werden\\
F-130 & Durch Gesten kann in der Vollbildansicht zum nächsten oder vorherigen Bild gewechselt werden\\
F-140 & Das gesamte UI ist für verschiedene Endgeräte (Tablets, Android TVs) optimiert\\
\hline
D-10 & Der Synchronisationlog und die Teilnehmerlisten werden in der Datenbank gespeichert\\
\hline
UI-30 & In der Hauptansicht eines Streams ermöglicht ein Menübutton die Teilnehmerliste einzusehen und Teilnehmer hinzuzufügen\\
UI-40 & Der Menüpunkt \enquote{Einstellungen} im Navigation Drawer ermöglicht die Synchronisationsmethode und allgemeine Synchronisationseinstellungen festzulegen\\
\end{longtable}
\end{center}